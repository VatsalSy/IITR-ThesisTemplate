\chapter{Numerical Framework}
\section{Introduction}
The collision of liquid jets has been studied in three-dimensional finite volume framework. Open source freeware, transient, multi-fluid, Navier-Stokes solver Gerris is used for the current study. Developed by \cite{Popinet2003,popinet2009}, Gerris has provided a stable and accurate platform for surface tension inclusive flows. It has been successful used frequently by researchers, such as \cite{chen2013high,kumar2016physical,kumar2017bending,kumar2017air}, to delve into similar problems in interfacial flows involving liquid sheets, jet and thin features like ligaments and films to capture intricate flow details and investigate the process. In this chapter, the detailed numerical framework adopted by Gerris is given. First, the governing equations and the corresponding discretization schemes are discussed followed by an illustration of the adaptive mesh refinement and grid independence study. At the end, a brief description about the work to validate the numerical model is presented.   
\section{Governing equations}
Conventional mass and momentum conservation equations for incompressible flow have been solved in presence of the surface tension and gravitational force. Equation~\ref{Equation::massConservation} contains the mass conservation equation for incompressible flow, which simply states that the velocity field ($V_i = V_1\hat{i} + V_2\hat{j} + V_3\hat{k}$) must be divergence free. 
\begin{equation}\label{Equation::massConservation}
\frac{\partial V_i}{\partial x_i} = 0
\end{equation}
The momentum equation for the incompressible Newtonian fluids that is solved for all the spatial coordinates can be summarized as given in equation~\ref{Equation::NS}. In the equation, the forces applied on the control volume chosen consist of the pressure in form of its gradient $\left(\frac{\partial P}{\partial x_i}\right)$, the volume specific body force due to gravitation $\left(\rho g_i\right)$, the surface forces due to shear stress ($2\mu D_{i,k}$, where $\mu$ represents the coefficient of dynamic viscosity and $D_{i,k} is the deformation tensor$) and the interface specific surface tension force ($\sigma \kappa$, where $\sigma$ is the surface tension coefficient and $\kappa$ denotes the curvature of the interface). 
\begin{equation}\label{Equation::NS}
\rho\left(\frac{\partial V_i}{\partial t} + V_k\frac{\partial V_i}{\partial x_k}\right) = -\frac{\partial P}{\partial x_i} + \frac{\partial(2\mu D_{i,k})}{\partial x_k} + \sigma\kappa\delta_sm_i + \rho g_i
\end{equation}
Moreover, the surface tension term is multiplied with the Dirac distribution function ($\delta_s$) to ensure that the force due to surface tension acts only at the interface having a normal vector $m_i$. Further, the deformation tensor $D_{i,k}$ is defined using the symmetric part of the velocity field gradient as given in equation~\ref{Equation::deformation}.
\begin{equation}\label{Equation::deformation}
D_{i,k} = \frac{1}{2}\left(\frac{\partial V_i}{\partial x_k} + \frac{\partial V_k}{\partial x_i}\right)
\end{equation}
The equation~\ref{Equation::NS} implicitly takes care of the mechanical energy. Moreover, the temperature variations are too small to affect the phenomenon being investigated and therefore, no thermal energy equation is considered. The interface tracking is done using the Volume Of Fluid (VOF), a front capturing approach involving volume fraction of liquid, defined as $\Psi(x_i, t)$, at the spatial and temporal instance of $x_i$ and $t$ respectively. The density and viscosity for the study can be described using equation~\ref{Equation::general} in terms of a generic property $A$.
\begin{equation} \label{Equation::general}
A (\Psi) = \Psi A_1 + (1-\Psi)A_2 \: \: \:  \forall  \: A \in \{\rho, \mu\}
\end{equation}
The VOF approach is implemented in a two-step process of interface reconstruction (based on the values of $\Psi$ and piecewise linear interface construction scheme, PLIC) along with geometric flux computation and interface advection, shown in equation~\ref{Equation::vof}.
\begin{equation} \label{Equation::vof}
\frac{\partial \Psi}{\partial t} + \frac{\partial(\Psi V_i)}{\partial X_i} = 0
\end{equation}
Gerris uses second-order accurate time discretization of momentum and continuity equations with time splitting algorithm as proposed by \cite{Chorin1968}, whereby an unconditionally stable corrector predictor time marching approach is adopted. A multigrid solver is used for the solution of the resulting pressure-velocity coupled Laplace equation. The advection term of the momentum equation $\left(V_k\frac{\partial V_i}{\partial X_k}\right)$ is estimated using the Bell-Colella-Glaz second-order unsplit upwind scheme \citep{bell1989second}, which requires the restriction to be set up on the time step. Following \cite{popinet2009}, time step has been determined to satisfy Courant-Friedrich-Lewy (CFL) stability criteria of less than unity. The details of solution procedure can be found in the works of \cite{Popinet2003,popinet2009}. In the next section, we have looked at the different process parameters which are relevant to this study followed by a grid independence study of the results.\\
 The density and viscosity field for the study can be defined as:
\begin{equation} \label{Equation::general}
A (\bar{\alpha}) = \bar{\alpha}A_1 + (1-\bar{\alpha})A_2 \: \: \:  \forall  \: A \in \{\rho, \mu\}
\end{equation}
In the Equation~\ref{Equation::general}, $\bar{\alpha}$ is calculated through bilinear interpolation of $\alpha(x_i,t)$ from cell centered values using spatial filtering. Mass conservation across the interface is implemented following Equation~\ref{Equation::mass}. 
\begin{equation} \label{Equation::mass}
\frac{\partial V_i}{\partial x_i} = 0
\end{equation}
 To tackle individual phases, a Volume of Fluid tracer advection has been solved, as described in Equation~\ref{Equation::vof}. 
\begin{equation} \label{Equation::vof}
\frac{\partial \alpha}{\partial t} + \frac{\partial(\alpha V_i)}{\partial x_i} = 0
\end{equation}
The temperature variations are assumed to be too small to affect the investigated phenomenon. As a result, no thermodynamic conservation equation is solved. However, the incompressible Navier-Stokes equations are solved for all three spatial coordinates. Equation~\ref{Equation::NS} has the representation for the same in Cartesian tensor notation. 
\begin{equation} \label{Equation::NS}
\rho\left( \frac{\partial V_i}{\partial t} + V_j\frac{\partial V_i}{\partial x_j} \right) = -\frac{\partial p}{\partial x_i} + \frac{\partial (2\mu D_{ij})}{\partial x_j} + \sigma \kappa \delta_sm_i + \rho g_i
\end{equation}
$\frac{\partial p}{\partial x_i}$ refers to the pressure gradient for the flow and $\rho g_i$ is the gravitational force applied at a given coordinate in space and time. Moreover, $\kappa$, $\sigma$ and $m_i$ are the curvature of the interface, surface tension force per unit length and the interface normal vector respectively. The multiplication of Dirac distribution function with the surface tension term is to ensure that this force is concentrated at the surface of the two phases. Further, the deformation tensor $D_{ij}$ is defined in Equation~\ref{Equation::deformation}.
\begin{equation} \label{Equation::deformation}
D_{ij} = \frac{1}{2}\left(\frac{\partial V_i}{\partial x_j} + \frac{\partial V_j}{\partial x_i}\right)
\end{equation}
Second order accurate time discretization of momentum and continuity equations are carried out with time splitting algorithm as proposed by Chorin \cite{Chorin1968}. Gerris decides time step based on Courant-Friedrichs-Lewy (CFL) stability criteria (CFL < 1) employed for Bell-Colella-Glaz second-order unsplit upwind scheme for the estimation of the velocity advection term, $\left(V_j\frac{\partial V_i}{\partial x_j} \right)$ \cite{popinet2009accurate}. However, a maximum limit of 5 X $10^{-4}$ seconds is set to keep the solution in bounds. \\

