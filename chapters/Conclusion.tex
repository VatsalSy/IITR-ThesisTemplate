\chapter{Conclusion and Future Work}
\section{Conclusion}
The stable chain structures are formed by the collision of laminar liquid jets when the inertia forces are, in order of magnitude, similar to the surface tension forces. A series of fully resolved detailed numerical simulations are performed to get the following conclusions:
\begin{enumerate}
\item [$\bullet$] The individual links, formed by collision of cylindrical jets (primary) or rims (secondary onward), occupy mutually orthogonal planes with a successive reduction in size owing to viscous effects. 
\item [$\bullet$] The variation of the velocity field across the radial coordinate is found to be negligible whereas the azimuthal variation of the sheet velocity is scaled using its average, given by an empiric relation. At the collision planes, the velocity field is found to be retracting in the direction of the colliding jets and rims whereas it is expanding in the plane of the formed sheets.
\item [$\bullet$] The inertial and gravitational forces provide a measure of the expansion of these sheets counteracted by the surface tension at the interface and viscous dissipations at the subsequent collisions. An increase in the impingement angle ($\alpha$) leads to wider links of the chain with a negligible change in the length of individual links. Intuitively, the size of the stable chain structure increases with an increase in the momenta of the jets ($Fr$) or with a decrease in the strength of the surface tension force (increasing $Bo$). Increase in the $Re$ presents a sharp increase in the dimensions of the chain, which saturates at the higher values of $Re$.
\item [$\bullet$] The individual symmetric sheet profile can be modeled using a third order polynomial, with an accuracy of $\pm$ 5\%, with coefficients dependent on various non-dimensional numbers featuring the interplay of different forces.
\item [$\bullet$] The analytical model is developed by considering the fundamental forces of gravity and surface tension lead to formulation of the sheet profile using only one free parameter which has been empirically related to the flow parameters.
\item [$\bullet$] Higher order links are found to be similar to lower or primary level element formed due to impact between jets of reduced $Fr$ and $\alpha$. 
\end{enumerate}
\section{Future Work}
\begin{enumerate}
\item [$\bullet$] Study of bubble entrainment by impingement of liquid jets into a pool surface and subsequent interactions between the clusters. 
\item [$\bullet$] Study of collision of liquid jets with high Reynolds number.
\end{enumerate}
\begin{table}[h]
\centering
\caption{Work Plan}
\label{my-label}
\begin{tabular}{@{}ccccc@{}}
\toprule
Objectives& \begin{tabular}[c]{@{}c@{}}May - \\ July 2017\end{tabular} & \begin{tabular}[c]{@{}c@{}}August - \\ October 2017\end{tabular}  & \begin{tabular}[c]{@{}c@{}}November 2017 - \\ January 2018\end{tabular}&  \begin{tabular}[c]{@{}c@{}}February - \\ April 2018\end{tabular}\\ \midrule
\begin{tabular}[c]{@{}c@{}}Formation of fluid chain by\\ collision of liquid jets\end{tabular}  &\cellcolor{gray}&\cellcolor{gray}&&\\
\begin{tabular}[c]{@{}c@{}}Interaction by impingement of\\  liquid jets into a pool\end{tabular} &&\cellcolor{gray}&\cellcolor{gray}&\\
\begin{tabular}[c]{@{}c@{}}Collision of liquid jets at\\ high Reynolds number\end{tabular}      &&&\cellcolor{gray}&\cellcolor{gray}\\ \bottomrule
\end{tabular}
\end{table}