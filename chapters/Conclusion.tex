\chapter{Conclusion and Scope of Future Work}
Combined experimental and numerical study has been performed to understand the dynamics of the interacting jets, and consequences of jet-pool impingement. Following categorical conclusion can be made from the present study.  
\section{Salient features and key findings}
\subsection{Collision of liquid jets}
The stable chain structures are formed by the collision of laminar liquid jets when the inertia forces are, in order of magnitude, similar to the surface tension forces. A series of fully resolved detailed numerical simulations are performed to get the following conclusions:
\begin{enumerate}
\item [$\bullet$] The individual links, formed by collision of cylindrical jets (primary) or rims (secondary onward), occupy mutually orthogonal planes with a successive reduction in size owing to viscous effects. 
\item [$\bullet$] The variation of the velocity field across the radial coordinate is found to be negligible whereas the azimuthal variation of the sheet velocity is scaled using its average, given by an empiric relation. At the collision planes, the velocity field is found to be retracting in the direction of the colliding jets and rims whereas it is expanding in the plane of the formed sheets.
\item [$\bullet$] The inertial and gravitational forces provide a measure of the expansion of these sheets counteracted by the surface tension at the interface and viscous dissipations at the subsequent collisions. An increase in the impingement angle ($\alpha$) leads to wider links of the chain with a negligible change in the length of individual links. Intuitively, the size of the stable chain structure increases with an increase in the momenta of the jets ($Fr$) or with a decrease in the strength of the surface tension force (increasing $Bo$). Increase in the $Re$ presents a sharp increase in the dimensions of the chain, which saturates at the higher values of $Re$.
\item [$\bullet$] The individual symmetric sheet profile can be modeled using a third order polynomial, with an accuracy of $\pm$ 5\%, with coefficients dependent on various non-dimensional numbers featuring the interplay of different forces.
\item [$\bullet$] An analytical model is developed by considering the fundamental forces of gravity and surface tension lead to the formulation of the sheet profile using only one free parameter which has been empirically related to the flow variables.
\item [$\bullet$] Higher order links are found to be similar to lower or primary level element formed due to impact between jets of reduced $Fr$ and $\alpha$. 
\end{enumerate}
\subsection{Air entrainment by impingement of liquid jet on a pool}
\begin{enumerate}
\item [$\bullet$] Entrainment of air bubbles due to impingement of liquid jet onto a pool has been modeled successfully using numerical simulations. This numerical model is validated with one to one correspondence with the experimental results. This cross-validation is done for the inception dynamics and pinch-off of the first annular bubble as well as the characteristics of bubble cluster. 
\item [$\bullet$] Depending on jet's strength, two opposite regimes have been identified as no entrainment and one with full-fledged bubble cluster. The latter is further categorized into two groups, namely, continuous entrainment with diverging-converging cluster for $Fr_DFr_L < 1.2$ and continuous entrainment with triangular cluster for $Fr_DFr_L > 1.2$.   
\item [$\bullet$] The inception of bubble entrainment consists of inertia influenced cavity formation and surface tension dominated collapse. 
\item [$\bullet$] The initial transients of bubble entrainment inception is tracked using high-speed imaging and numerical simulations. A quasi-steady state of the process is reached which has been used to study bubble cluster characteristics. 
\item [$\bullet$] With R-squared values of $0.9$ and $0.95$, the entrained cluster depth of penetration is co-related with the inertial strength of liquid jet for two distinct full-fledged entrainment cluster regimes.
\item [$\bullet$] Centroid of the bubble cluster is tracked using in-house image processing code. The spatio-temporal variation of centroid location leads to a linear relationship between its acceleration and its deviation from the plane of jet impact. This accounts for the observed oscillations of the bubble cluster, whose time-period increases with the inertial strength of the liquid jet. 
\item [$\bullet$] Trajectory of a single bubble has been also studied to establish formation, downward traverse, stabilized floating at critical height, buoyancy-driven approach towards free surface and collapse in its whole life cycle.
\end{enumerate}
\section{Scope of future work}
After a detailed study of the interaction between symmetric jets for the formation of a liquid chain-like structure and normal impingement of jet in a pool following topics can be identified as potential directions for future research:
\begin{enumerate}
\item Dynamics of jet interactions at higher Re may lead to the flapping of sheet, fragmentation, and atomization. A detailed experimental and numerical study can be performed to understand these rich fluidic physics.
\item Moreover, impingement between two asymmetric jets can lead towards complex liquid structures and such occurrences are quite relevant for industrial applications. A detailed study can be targeted towards that direction.
\item It will be also interesting to see the interaction between jet and sheet or sheets to have a three-dimensional mesmerizing fluidic structure. Study of such topics will be quite useful for batch processing of different chemical reactors and spray jet cooling technique.
\item Impingement of inclined jets in a pool can be readily observed in impingement cooling systems which will be an extension of present understanding of orthogonal jet impingement in a pool. A natural extension of the present work can be an experimental and numerical observation of entrained cluster under inclined impingement.
\item Cluster-cluster interaction during side by side impingement of jets in a pool can be also studied from the same setup developed in the present study. It will give insights towards population interactions between bubbles swarms.
\item Last but not the least, the bubble entrainment cluster penetrating through multiple layer liquid stratification may lead to the presence of complex multiphase situations. The study will be quite rich in fluid dynamics as the interaction between different scale and phases are involved. 
\end{enumerate}
