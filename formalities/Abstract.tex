\begin{abstract}
\addchaptertocentry{\abstractname} % Add the abstract to the table of contents
Formation of liquid chain is studied through a series of fully resolved detailed numerical simulations. The collision of liquid jets and formation of a sheet in the median plane is illustrated numerically. The sheet subsequently transforms into a chain like fluidic structure with successive dwarf links in mutually orthogonal planes. This structure is analyzed to pertain towards a steady state. Since the velocity of the sheet is super-critical (higher than the speed of the capillary waves), the first link is studied for the flow kinematics. To understand the behavior of fluid parcels inside the chain, flow is studied with streamlines. Their radial dispatch after the collision of jets is followed by self-similar paths with respect to the chain outer periphery. Further, a scaled law is presented for the variation of fluid velocity across the azimuthal direction of the flow. The influence of several non-dimensional parameters has been found on the first link of the chain. This has been generalized over the entire chain structure. For the understanding of chain profiles over a wide range of operating parameters, a correlation has been proposed based on numerical simulations and subsequent regression analysis. Citing analogy between the impact of jets for the formation of elemental links and traversal of non-deformable fluid quanta after the collision, an attempt has been made to understand the fundamental physics of this phenomenon through force balance. The analogy helps to take into account the role of surface tension and other forces on the shape and size of the liquid sheets. Further, the formation of higher order links is proposed as equivalent to the collision between the liquid rims bounding the sheet, modeled as the jets of reduced strengths and smaller impingement angles. Finally, we assess the effects of various fluid properties on the dimensions of these links, illustrating the viscous dissipation at the time of collisions.  
\end{abstract}