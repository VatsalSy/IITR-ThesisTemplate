\begin{abstract}
\addchaptertocentry{\abstractname} % Add the abstract to the table of contents
Interactions between two facing liquid jets and consequences of jet impingement onto a pool are studied using detailed numerical simulations and intelligent experimental investigations. The collision between liquid jets leads to the formation of a sheet in the median plane and is illustrated numerically. The sheet subsequently transforms into a chain like fluidic structure with successive dwarf links in mutually orthogonal planes. This structure is analyzed to pertain towards a steady state. Since the velocity of the sheet is super-critical (higher than the speed of the capillary waves), the first link is studied for the flow kinematics. To understand the behavior of fluid parcels inside the chain, flow is studied with streamlines. Their radial dispatch after the collision of jets is followed by self-similar paths with respect to the chain outer periphery. Further, a scaling law is presented for the variation of fluid velocity across the azimuthal direction of the flow. The influence of several non-dimensional parameters has been found on the shape of the first link of the chain. This has been generalized over the entire chain structure. For the understanding of chain profiles over a wide range of operating parameters, a correlation has been proposed based on numerical simulations and subsequent regression analysis. Citing analogy between the impact of jets for the formation of elemental links and traversal of non-deformable fluid quanta after the collision, an attempt has been made to understand the fundamental physics of this phenomenon through force balance. The analogy helps to take into account the role of surface tension and other forces on the shape and size of the liquid sheets. Further, the formation of higher order links is proposed as equivalent to the collision between the liquid rims bounding the sheet, modeled as the jets of reduced strengths and smaller impingement angles. Finally, the effects of various fluid properties on the dimensions of these links are assessed, illustrating the viscous dissipation at the time of collisions. \\
Furthermore, the air entrainment due to impingement of a water jet on a pool is studied extensively to understand the physics of the initiation and the cluster of bubbles formed below the free surface. Possible outcomes due to the jet impingement in a pool have been identified as smooth free surface without entrainment or formation of rigorous bubble cluster below the jet-pool contact. A triangular entrained region is found to be a three-dimensional association of disconnected bubble population continuously breaking and making with the neighbors. A correlation for prediction of maximum entrained height for a range of jet diameters and lengths is proposed. The trajectory of a single bubble is also studied to understand the kinematics of the bubble cluster. Alongside, an electrical conductivity probe has been used to examine the probabilistic presence of the bubble at a given depth in the liquid pool. 
\end{abstract}